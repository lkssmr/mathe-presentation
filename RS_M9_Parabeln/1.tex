%\documentclass[10pt]{beamer} % 4/3 aspect ration works as well
\documentclass[10pt,aspectratio=169]{beamer}
%----------------------------------------------------------------------------------------
%	PACKAGES AND THEMES
%----------------------------------------------------------------------------------------

\mode<presentation>{
  \usetheme{EuXFEL}
  \usecolortheme{EuXFEL}
}
\usepackage{hyperref}
\newcommand{\la}{\lambda}
\newcommand{\dl}{\Delta \lambda}
\newcommand{\Ra}{\Rightarrow}
\usepackage[utf8]{inputenc}
\usepackage[T1]{fontenc}
\usepackage{pgfplots}
\usepackage{xcolor}
\usepackage{chngcntr}
\usepackage{amsmath}
\usepackage{mathrsfs}
\usetikzlibrary{arrows}
\usetikzlibrary[patterns]
\usepackage{mathtools}
\pgfplotsset{compat=1.15}
\usepackage{amssymb}
\usepackage{amsthm}
\usepackage{geometry}
%\geometry{tmargin=25mm, bmargin=20mm, lmargin=50mm, rmargin=35mm}
\usepackage{mathtools}
\usepackage{drawmatrix}
%\theoremstyle{defintion}
\newtheorem {satz}{Satz}
\usepackage{pdfpages}
\bibliographystyle{abbrv}
\usepackage{lastpage}
\usepackage[fixFPpow]{tabularcalc}
\usepackage{xcolor}
\newtheorem{geg}{Gegeben}
\usepackage{graphicx}
\usepackage{hyperref}
\graphicspath{Bilder/}
\usepackage{chngcntr}

\usepackage{mathrsfs}

\renewcommand{\d}{\text{d}}
\newcommand{\Mod}[1]{\ (\mathrm{mod}\ #1)}
\newcommand{\R}{\mathbb{R}}
\newcommand{\N}{\mathbb{N}}
\newcommand{\dx}{\; \text{d} x}
\newcommand{\x}{\cdot}
\newcommand{\red}{\color{red}}
\newcommand{\green}{\color{olive}}
\newcommand{\blue}{\color{blue}}
\newcommand{\black}{\color{black}}
\newcommand{\nt}{\notag}
\newcommand{\dis}{b^2-4ac}
\newcommand{\trennl}{\noindent\rule[0.5ex]{\linewidth}{1.5pt}}
\newcommand{\btrennl}{\noindent\rule[0.5ex]{\linewidth}{3pt}}
\newcommand{\abc}{\frac{-b \pm \sqrt{b^2-4ac}}{2a}}
\usepackage{multicol}
\usepackage{mdframed}

\theoremstyle{definition}
\newtheorem{losung}{Lösung}[section]

\theoremstyle{definition}
\newtheorem{defi}{Definition}[section]

\theoremstyle{definition}
\newtheorem{ad}{Auf Deutsch:}[section]


\theoremstyle{definition}
\newtheorem{aufgabe}{Aufgabe}[section]

\usepackage{url}

\theoremstyle{remark}
\newtheorem{anm}{Anmerkung}[section]

\theoremstyle{definition}
\newtheorem{mathem}{mathematisch augedrückt} 

\renewcommand{\theequation}{\arabic{section}.\arabic{equation}}
\renewcommand{\qedsymbol}{$\blacksquare$}

%\titleformat{\section}
%{\normalfont \bfseries }{\thesection.}{}{}

%\counterwithin*{equation}{page}

\usefonttheme[onlymath]{serif}

\usepackage{lmodern}
\usepackage[ngerman]{babel}
\usepackage{verbatim}
\usepackage{datetime}

\usepackage[scaled]{roboto}
\setbeamertemplate{theorems}[numbered]
\setbeamertemplate{caption}[numbered]

\renewcommand{\familydefault}{\sfdefault}
\renewcommand\mathfamilydefault{}
\newcommand{\colorbf}[1]{{\color{xOrange}\textbf{#1}}}

\newdateformat{dmydate}{%
    \twodigit{\THEDAY}~\monthname[\THEMONTH] \THEYEAR
%    \dayofweekname{\THEDAY}{\THEMONTH}{\THEYEAR} \twodigit{\THEDAY}~\monthname[\THEMONTH] \THEYEAR      
}   

\title[Parabeln]{Parabeln und Quadratische Funktionen}
\subtitle[Schnitte]{Schnittprobleme}

\author{Lukas Semrau } % Your name
\institute[] % Your institution as it will appear on the bottom of every slide, may be shorthand to save space
{\noindent
  \textit{lukas@lukas-semrau.de}
}
\date{\dmydate\today} % Date, can be changed to a custom date

\setbeamersize{text margin left=.05\pdfpagewidth,text margin right=.05\pdfpagewidth}

%\vfill
%\includegraphics[scale=0.55,keepaspectratio]{desy_logo.pdf}


\begin{document}

{
  \setbeamertemplate{headline}{}
  \begin{frame}
    \titlepage
   \end{frame}
}
\section{Allgemeines}
\begin{frame}{Was ist eine quadr. Funktion}
    \begin{defi}
    Eine quadratische Funktion hat die Form \begin{align}
        f(x)=y=ax^2+bx+c,
    \end{align} wobei in meisten $a=1$ gilt.
    \end{defi}
    \pause
    Für eine Funktion $h(x)=x^2-4x+3$ gelten folgende Parameter: \pause
    \begin{align*}
        a=1;\quad b = -4; \quad c = 3
    \end{align*}
\end{frame}
\begin{frame}{Was ist eine Parabel?}
\begin{defi}
    \begin{itemize}
        \item Graph d. quadratischen Funktion \pause
        \item Für $f(x)=x^2$ heißt der Graph \textbf{Normalparabel.}
    \end{itemize}
\end{defi}
    \pause 
    \begin{figure}
    \centering
    \begin{tikzpicture}
\begin{axis}[
x=0.5cm,y=0.5cm,
axis lines=middle,
xmin=-2.5,
xmax=2.5,
ymin=-1,
ymax=4.5,
%xtick={1,2},
%xticklabels={$x_0$,$x_0+\delta$},
%yticklabels={$f(x_0)$,$f(x_0+\delta)$},
]
%Below the red parabola is defined

\addplot [
    domain=-2:2, 
    samples=100, 
    color=red,
]{x^2};


\end{axis}
\end{tikzpicture}
    \caption{Schnitte zweier Parabeln}
    \label{fig:schnitt}
\end{figure}
\end{frame}

\begin{frame}{Scheitelpunkt}
    \begin{defi}
    Der unterste höchste Punkt heißt Scheitelpunkt $S(x_E \mid y_E)$ mit $x_E$ als Extremstelle.
    \end{defi} \pause
        \begin{figure}
    \centering
    \begin{tikzpicture}
\begin{axis}[
x=0.5cm,y=0.5cm,
axis lines=middle,
xmin=-3.5,
xmax=3.5,
ymin=-2,
ymax=2,
%xtick={1,2},
%xticklabels={$x_0$,$x_0+\delta$},
%yticklabels={$f(x_0)$,$f(x_0+\delta)$},
]
%Below the red parabola is defined

\addplot [
    domain=-2:2, 
    samples=100, 
    color=red,
]{x*(x-2)};
\addplot [
    domain=-2:2, 
    samples=100, 
    color=blue,
]{-1*(x+1)^2+1};


\end{axis}
\end{tikzpicture}
    \caption{Schnitte zweier Parabeln}
    \label{fig:schnitt}
\end{figure}
    
\end{frame}

\begin{frame}{Scheitelpunktform}
    \begin{defi}
    Der Scheitel einer Funktion \begin{align}
        y = (x-d)^2+e \qquad \mid \text{Scheitelpunktform}
    \end{align} liegt bei $S(d\mid e)$.
    \end{defi}
\end{frame}
\begin{frame}{Von der Normalform zur Scheitelpunktform}
    \begin{align}
    \begin{aligned}
    y = x^2+bx+c &= x^2+ 2\x x \x \frac{b}{2} + \left( \frac{b}{2} \right)^2- \left( \frac{b}{2} \right)^2+c\\ 
    &= \left( x+\frac{b}{2} \right)^2-\left( \frac{b}{2} \right)^2+c \\
    x_E &= -\frac{b}{2}
    \end{aligned}
    \end{align}
\end{frame}
\begin{frame}{Lösungsformel}
    Für die Gleichung $x^2+bx+c=0$ gilt für $x$:
    \begin{align}
        x&=\frac{-b\pm \sqrt{b^2-4c}}{2} \\
        \text{oder} \qquad &= -\frac{b}{2} \pm \sqrt{\left(\frac{b}{2} \right)^2-c}
    \end{align}
\end{frame}
\begin{frame}{Schnittpunkte zweier Parabeln}
\begin{figure}
    \centering
    \begin{tikzpicture}
\begin{axis}[
 title={\textit{Schnittpunkte zweier Parabeln}},
 ymajorgrids=true,
xmajorgrids=true,
x=1cm,y=1cm,
axis lines=middle,
xmin=-2.5,
xmax=3.5,
ymin=-2.5,
ymax=1.5,
%xtick={1,2},
%xticklabels={$x_0$,$x_0+\delta$},
%yticklabels={$f(x_0)$,$f(x_0+\delta)$},
]
%Below the red parabola is defined

\addplot [
    domain=-2:3, 
    samples=100, 
    color=red,
]{-x^2};
\addlegendentry{$f$};
\addplot [
    domain=-2:3, 
    samples=100, 
    color=blue,
]{x*(x-2)};
\addlegendentry{$h$};
\draw[fill=black] (0,0) circle (1pt);
\draw[fill=black] (1,-1) circle (1pt);
\draw[dashed] (0,-1) -- (1,-1) -- (1,0); 

\end{axis}
\end{tikzpicture}
    \caption{Schnitte zweier Parabeln}
    \label{fig:schnitt}
\end{figure}
\end{frame}
\begin{frame}{Schnittpunkte}
\begin{figure}
    \centering
    \begin{tikzpicture}
\begin{axis}[
ymajorgrids=true,
xmajorgrids=true,
 title={\textit{Schnittpunkte}},
x=1cm,y=1cm,
axis lines=middle,
xmin=-2.5,
xmax=2.5,
ymin=-2.5,
ymax=2.5,
%xtick={1,2},
%xticklabels={$x_0$,$x_0+\delta$},
%yticklabels={$f(x_0)$,$f(x_0+\delta)$},
]
%Below the red parabola is defined

\addplot [
    domain=-2:3, 
    samples=100, 
    color=red,
]{x};
\addlegendentry{$f$};
\addplot [
    domain=0.1:3, 
    samples=100, 
    color=blue,
]{1/x};
\addplot [
    domain=-3:-0.1, 
    samples=100, 
    color=blue,
]{1/x};
\addlegendentry{$h$};

\end{axis}
\end{tikzpicture}
    \caption{Schnitte von Hyperbel und Gerade}
    \label{fig:schnitt1}
\end{figure}
\end{frame}
\begin{frame}{Schnittpunkte}
\begin{figure}
    \centering
    \begin{tikzpicture}
\begin{axis}[
ymajorgrids=true,
xmajorgrids=true,
 title={\textit{Schnittpunkte}},
x=1cm,y=1cm,
axis lines=middle,
xmin=-2.5,
xmax=2.5,
ymin=-2.5,
ymax=2.5,
%xtick={1,2},
%xticklabels={$x_0$,$x_0+\delta$},
%yticklabels={$f(x_0)$,$f(x_0+\delta)$},
]
%Below the red parabola is defined

\addplot [
    domain=-2:3, 
    samples=100, 
    color=red,
]{x};
\addlegendentry{$f$};
\addplot [
    domain=-3:3, 
    samples=100, 
    color=blue,
]{x*(x+1)-1};
\addlegendentry{$h$};

\end{axis}
\end{tikzpicture}
    \caption{Schnitte von Parabel und Gerade}
    \label{fig:schnitt1}
\end{figure}
\end{frame}

\end{document}